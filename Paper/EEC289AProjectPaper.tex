\documentclass[conference]{IEEEtran}
\usepackage{amsmath,amssymb,amsfonts}
\usepackage{algorithmic}
\usepackage{algorithm}
\usepackage{array}
\usepackage[caption=false,font=normalsize,labelfont=sf,textfont=sf]{subfig}
\usepackage{textcomp}
\usepackage{stfloats}
\usepackage{url}
\usepackage{verbatim}
\usepackage{graphicx}
\usepackage{xcolor}
\usepackage{cite}

\graphicspath{{figures/}} %set folder for figures

\begin{document}

\title{Anomaly Detection for Time Series Data using VAE-LSTM Model}

\author{
    \IEEEauthorblockN{Randall Fowler}
    \IEEEauthorblockA{\textit{Department of ECE} \\
    \textit{UC Davis}\\
    rlfowler@ucdavis.edu}
    \and
    \IEEEauthorblockN{Conor King}
    \IEEEauthorblockA{\textit{Department of ECE} \\
    \textit{UC Davis}\\
    cfking@ucdavis.edu}
    \and
    \IEEEauthorblockN{Ajay Suresh}
    \IEEEauthorblockA{\textit{Department of ECE} \\
    \textit{UC Davis}\\
    ajsuresh@ucdavis.edu}
}

\maketitle

\begin{abstract}
Here is a abstract...
\end{abstract}

\begin{IEEEkeywords}
Here are some keywords...
\end{IEEEkeywords}

\section{Introduction}
\IEEEPARstart{T}{his} is an introduction...

\section{Methodology}
Anomaly detection is represented by measuring the reconstruction error from a given deep learning model. For this system, a VAE and LSTM hybrid model is used for finding a latent space with a known distribution and capturing temporal information over time series data. This data is presented in a windowed fashion to locate anomalies within a given window. Evaluation of the anomaly detection will be measured using the harmonic mean of precision and recall (F1 score).

\subsection{Preprocessing}
The nature of the data used for this detection system does not need to be specific, but it does require being time series data. With a given window size, a sliding window will sample the original signal, and the VAE input and output will be equivalent to the size of the window. 

As this data does not need to be labeled, it does need to be clean of noise or anomalies. This is due to the model learning the clean distribution of the data, and after training, the poor reconstruction of a signal will indicate the possibility of an anomaly.


\subsection{Variational Autoencoder}
An autoencoder structure establishes a method for compressing information into a dense latent space. This space is not unique as the original signal can be encoded into a large number of different latent spaces. The VAE addresses this non-regularization issue of the latent space by forcing the space to follow a specific distribution. For most VAEs, this distribution is a normal distribution, and the output of the encoder will be a prediction of mean and standard deviation. Once the distribution of the latent space is known, a sampling will be performed to create a latent vector for the input of the decoder. When the latent space distribution is well formed, the output of the decoder aims to reconstruct the original signal as best as possible.


\subsection{Long Short-Term Memory}
Here is the LSTM...

\subsection{VAE-LSTM Model Training}
Here is the VAE-LSTM model training...

\subsection{Anomaly Detection}
Here is the anomaly detection...

\subsection{Evaluation Metrics}
Here are the evaluation metrics...


\section{Experiments}
Here are the experiments...

\subsection{Inertial Measurement Unit Dataset}
Here is the dataset...

\subsection{Synthesized Photoplethysmography Dataset}
Here is the dataset...

\subsection{Electric Vehicle Drive Cycle Dataset}
Here is the dataset...


\section{Results}
Here are the results...

\subsection{Inertial Measurement Unit}
Here is the dataset...

\subsection{Synthesized Photoplethysmography}
Here is the dataset...

\subsection{Electric Vehicle Drive Cycle}
Here is the dataset...


\section{Conclusion}% a little of last page
Here is a conclusion...

\subsection{Future Work}
Here is some future work...

\section*{Acknowledgment}
The authors would like to thank Dr. Yubei Chen for his guidance and support on this course project.

\section*{Contributions}
Randall Fowler developed the models, evaluation methods, and experimented with the IMU dataset. Conor King experimented with the synthesized PPG dataset. Ajay Suresh experimented with the EV drive cycle dataset. All authors contributed to the writing of the paper.

\begin{thebibliography}{1}
\bibliographystyle{IEEEtran}

\bibitem{ref1}
Citation 1

\bibitem{ref2}
Citation 2

\bibitem{ref3}
Citation 3

\bibitem{ref4}
Citation 4

\bibitem{ref5}
Citation 5

\end{thebibliography}

\end{document}